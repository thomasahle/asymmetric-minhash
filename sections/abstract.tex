%! TEX root = ../aminhash.tex

%\title{Getting More from MinHash}
%\title{Improved MinHash Estimation for Nearest Neighbor Search}
%\title{Minner: Improved MinHash Estimation by Counting the Values less than the Minimum}
%\title{Minner: Improved Recovery of Set Similarity in MinHash Quantized Databases by Counting the Values less than the Minimum}
\title{Minner: Improved Similarity Estimation and Recall on MinHashed Databases}

\begin{abstract}
   Quantization is the state of the art approach to efficiently storing and searching large high-dimensional datasets.
   Broder'97~\cite{broder1997resemblance} introduced the idea of Minwise Hashing (MinHash) for quantizing or sketching large sets or binary strings into a small number of values, and provided a way to reconstruct the overlap or Jaccard Similarity between two sets sketched this way.

   In this paper, we propose a new estimator for MinHash in the case where the database is quantized, but the query is not.
   By computing the similarity between a set and a MinHash sketch directly, rather than first also sketching the query, we increase precision and improve recall.

   We take a principled approach based on maximum likelihood (MLE) with strong theoretical guarantees.
   Experimental results show this reduces the number of MinHash values needed by 20-30\% for equivalent recall.
   Finally we suggest a third estimator which is as fast as the classical MinHash estimator while nearly as precise as the MLE.


   %and derive a fast practical algorithm with 
   %We test the estimator on standard datasets and obtain smaller reconstruction error and higher recall for nearest neighbour search.

   %Our methods touch only on the query side, allowing application to databases already quantized by Minwise Hashing.

   %Where estimating similarity from two sketches corresponds to the third party model of communication complexity, our setting corresponds to standard one way complexity, for which we give a new upper bound on the necessary bits to obtain a given variance of estimation.

   % We can also compute the global risk of the MLE estimator, which is $0.178847/K$, $28.5\%$ less than the Classical Estimator.


%   We introduce a new way to estimate the similarity $J(X,Y)$ given $h(Y)$ and $X$, that is asymptotically optimal in information theoretical terms, and which in practice reduces the number of samples needed by 30\% for a given recall.
%
%
%   We save a factor of $>1.42$ in variance over the standard method,
%   which means a $16\%$ saving in the necessary number of MinHash samples for equivalent confidence bounds.
%   In the case of asymmetric set sizes and large similarities the improvement is particularly good, and we show an experimental improvement in recall on standard binary datasets.
\end{abstract}

