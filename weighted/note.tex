\documentclass[11pt]{article}
\usepackage[utf8]{inputenc}
\usepackage[UKenglish]{babel}

\usepackage{amsmath}
\usepackage{amsthm}
\usepackage{amsfonts}
\usepackage{amssymb}

\newtheorem{theorem}{Theorem}
\newtheorem{lemma}{Lemma}
\newtheorem{corollary}{Corollary}
\newtheorem{definition}{Definition}
\newtheorem{example}{Example}

% Thomas packages

\usepackage{algorithm}
\usepackage[noend]{algpseudocode}
\usepackage{enumitem}
\usepackage{multicol}
\usepackage{hyperref}
\usepackage{cleveref}
\let\globcount\newcount % https://tex.stackexchange.com/questions/285950/package-autonum-needs-the-obsolete-etex-package
\usepackage{autonum}
\DeclareMathAlphabet{\mathpzc}{OT1}{pzc}{m}{it}

\DeclareMathOperator{\poly}{poly}
\DeclareMathOperator{\sign}{sign}
\DeclareMathOperator{\asvec}{vec}
\DeclareMathOperator*{\argmax}{arg\,max}
\DeclareMathOperator*{\argmin}{arg\,min}
% Allow page break in align
\allowdisplaybreaks


\usepackage{mathtools}
\DeclarePairedDelimiterX{\infdivx}[2]{(}{)}{#1 \,\delimsize\|\, #2}
\DeclareMathOperator*{\Div}{D}
\newcommand{\D}{\Div\infdivx}
\DeclareMathOperator*{\smallDiv}{d}
\newcommand{\di}{\smallDiv\infdivx}
\DeclareMathOperator*{\h}{h}

% Jakob packages
\newcommand{\eps}{\varepsilon}
\newcommand{\R}{\mathbb{R}}
\newcommand{\Z}{\mathbb{Z}}

\newcommand{\abs}[1]{\left\lvert #1\right\rvert}
\newcommand{\set}[1]{\left\{ #1 \right\}}
\newcommand{\setbuilder}[2]{\left\{ #1 \; \middle\vert \; #2 \right\}}
\newcommand{\norm}[2]{\left\| #1 \right\|_{#2}}
\newcommand{\innerprod}[2]{\left\langle #1, \; #2 \right\rangle}

\newcommand{\Prp}[1]{\Pr\!\left[#1 \right]}
\newcommand{\Prpcond}[2]{\Pr\!\left[#1 \; \middle| \; #2 \right]}
\DeclareMathOperator*{\Ep}{E}
\newcommand{\Epcond}[2]{\text{E}\!\left[#1 \; \middle| \; #2 \right]}

\newcommand{\br}[1]{\left( #1 \right)}


\newcommand{\smat}[1]{\left[\begin{smallmatrix}#1\end{smallmatrix}\right]}
\newcommand{\svec}[1]{[\begin{smallmatrix}#1\end{smallmatrix}]}

\usepackage{csquotes}

\begin{document}

Assume $X,Y\subseteq U$ are sampled independently by picking element $i\in [u]$ with probability $p_i$.
We condition on $|X|=n_x$, $|Y|=n_y$ and $|X\cap Y|=v$,
e.g. by rejection sampling.
Now assume the elements of $[u]$ are ordered in some way.
We write down the first $k$ elements of $Y$ under this ordering.
Call this subset $Y'$.
Now the goal is to estimate $v$ given $X$, $Y'$ and $n_y$.

We build a maximum likelihood estimator.
Let $s$ be the largest value of $Y'$.
(Note we use 1-indexing $[u]=\{1,\dots,u\}$.)
Then we may equivalently see $Y' = Y \cap [s]$.
Define $m=|X\cap[s]$ and $c=|X\cap Y\cap [s]| = |X\cap Y'|$.
Define $Z_i = [i\in Y]$ and $Z_T = \sum_{i\in T} Z_i$.
Then
\[
   \Pr[Y' | v]
   = \frac{
      \Pr[Z_{X\setminus [s]}=v-c]
      \Pr[Z_{[u]\setminus [s]\setminus X}=n_y-k-(v-c)]
   }{
      \Pr[Z_{X}=v]
      \Pr[Z_{[u]\setminus X}=n_y-v]
   }
   .
\]

To analyze this, define the cumulant generating function
\[
   \kappa_T(t)
   = \log E[\exp(t Z_T)]
   = \sum_{i\in T} \log[(1-p_i) + p_i \exp(t)]
   .
\]
Then
\[
   \log\Pr[Z_{T}=v] \approx \kappa_T(t) - t v,
   \label{eq:logprob}
\]
where $t$ is defined by $\kappa'_T(t) = v$, (recall $\kappa'_T(t) = \sum_{i\in T}\frac{1}{1+\frac{1-p_i}{p_i}e^{-t}}$.)
This means the log-like function can be written as
\begin{align}
   \ell(v) = \log\Pr[Y' | v]
   &= 
   \kappa_{X\setminus [s]}(t_1)-t_1(v-c)
   +\kappa_{[u]\setminus [s]\setminus X}(t_2)-t_2(n_y-k-v+c)
   \\
   &-\kappa_{X}(t_3)+t_3v
   -\kappa_{[u]\setminus X}(t_4)+t_4(n_y-v)
   ,
\end{align}
where $\kappa_{X\setminus [s]}(t_1) = v-c$ and so on.
It is useful to note that in the setting of \cref{eq:logprob}
we have $\frac{d}{dv}\kappa_T(t)-tv = \kappa_T'(t)t' - t'v - t = -t$.
Thus
\[
   \frac{d}{dv}\ell(v) = -t_1+t_2+t_3-t_4.
   \label{eq:d1}
\]
An alternative interpretation of the MLE is thus to solve $t_1+t_4=t_2+t_3$ over $v$.

We are interested in the variance of the estimator as a function of $u, n_x, n_y, v$ and the sequence $p_i$.
In particular we want to know which ordering of the $p_i$ will minimize this variance.
We thus consider the Fischer information
\[
   I(v) = -E[\frac{d^2}{dv^2}\ell(v)].
\]
Note that $t=(\kappa_T')^{-1}(v)$ so we can compute $\frac{d}{dv}t = \frac{1}{k_T''((\kappa_T')^{-1}(v))} = \frac{1}{k_T''(t)}$.
Applying this to \cref{eq:d1} we find
\[
   \frac{d^2}{dv^2}\ell(v)
   =
   -\frac{1}{\kappa_{X\setminus [s]}''(t_1)}
   -\frac{1}{\kappa_{[u]\setminus [s] \setminus X}''(t_2)}
   +\frac{1}{\kappa_{X}''(t_3)}
   +\frac{1}{\kappa_{[u]\setminus X}''(t_4)}.
\]
(Note some of the signs changed due to the sign of $v$ in the defining equations.)

In the case of normally distributed random variables we would assume
$\kappa_T''(t) = \sigma_T^2 = \sum_{i\in T} p_i(1-p_i)$.
We might then approximate $E[1/\kappa''_T(t)]$ by $1/E[\sigma_T^2]$.
Here
\[
   E[\sigma_T^2] = \sum_{i\in [u]}p_i(1-p_i)\Pr[i\in T].
\]

\begin{displayquote}

   Jakob suggests another approach:
   Let $\kappa_\theta$ be the cumulant generating function for the $\theta$ tilted distribution of $Z$.
   Then $\kappa_\theta(t) = \kappa(t+\theta) - \kappa(\theta)$ and $\kappa_\theta''(t) = \kappa''(t+\theta)$.
   Taking $t=0$ we get $\kappa''(\theta) = \kappa_\theta''(0) = \sigma_\theta^2$, which is just the variance of the tilted distribution.

   Note that
   \begin{align}
      \Pr[Z=k]e^{tk}/m(t)
      &= \sum_{|I|=k}\prod_{i\in I}p_i\prod_{i\not\in I}(1-p_i) e^{tk}/\prod_{i\in[n]}m_i(t)
    \\&= \sum_{|I|=k}\prod_{i\in I}\frac{p_ie^t}{1-p_i+p_i e^t}\prod_{i\not\in I}\frac{1-p_i}{1-p_i+p_ie^t}
    \\&= \sum_{|I|=k}\prod_{i\in I}\frac{p_ie^t}{1-p_i+p_i e^t}\prod_{i\not\in I}\left(1-\frac{p_i e^t}{1-p_i+p_ie^t}\right).
   \end{align}
   In other words, the tilted sum $Z$ is just the sum of individually tilted Bernoulli random variables.
   Thus
   \[
      \kappa''(t)
      = \kappa''_t(0)
      = \sigma_t^2
      = \sum_{i\in T} \frac{p_i(1-p_i)}{(1-p_i+p_i e^t)^2}e^t
      .
   \]
   Actually a much easier approach would be to just note $\kappa(t) = \sum_i \kappa_i(t)$ so $\kappa''(t) = \sum_i \kappa_i''(t)$, which is exactly the above.

\end{displayquote}

Would it be easier to take the expectation of $t$ first and then the second derivative in $v$?
I think that's allowed...
The expectation of $t^2$ would also suffice, no subsequent derivatives needed.

This seems hard.
Let's go back to just
\[
   E[\sigma_X^2]
   = \sum_{i\in[u]}p_i(1-p_i)\Pr[i\in X]
   = \sum_{i\in[u]}p_i^2(1-p_i).
\]
Similarly 
\[
   E[\sigma_{X\setminus[s]}^2]
   = \sum_{i\in[u]}p_i(1-p_i)\Pr[i\in X \wedge i\not\in [s]]
   = \sum_{i > s}p_i^2(1-p_i)
\]
Now to maximize the Fischer Information we want to maximize
\[
   \frac1{\sum_{i > s}p_i^2(1-p_i)}
   - 
   \frac{1}{\sum_{i\in[u]}p_i^2(1-p_i)}
   .
\]
Note this quantity is clearly non-negative.

Experiments suggest that a reasonable choice may be split sequence like [5, 4, 3, 2, 1, 0, 6, 7, 8, 9].
That is we start with the middle values and move down, then come back and take the remaining in increasing order.

For probabilities like $[.1, .2, .3, \dots]$ the sorted order appears good,
where as for parreto probabilities the inverse order may be best.
Of course, there is something about the mean that's a little odd.
With parreto the sum will always be < 1?

%Perhaps a reasonable choice is having 

%Idea: Can we at least use that $\kappa''(t)$ seems to be convex in $p$?
%Can we pick some real values and see which ordering of $p$ improves $k''(t)$ for those?
%Hm, the problem is that it wouldn't be an average over $s,c,m$ and sampling those also depends on the ordering of $p$.

Ok, so my program now indicates that increasing order is optimal for entropy maximisation.
That is, let $S\cap Y$ be the observed bottom-$k$ sketch, then
\[
   H(S\cap Y) = E_{S,Y} \log\Pr[S\cap Y]
\]
is maximized when $p$ is increasing.
In fact, it doesn't even matter if we include the case where $|Y|$ is smaller than $k$.
It seems.

In particular we look at the sum
\[
   \sum_{s, Y'\subseteq[s], |Y'|=k-1}
   p_s p^{Y'} (1-p)^{\overline{Y'}}
   \log\frac{1}{
      p^{Y'} (1-p)^{\overline{Y'}}
   }
\]
where $p(T)=\prod_{i\in T} p_i \prod_{i\not\in T}(1-p_i)$

\end{document}
